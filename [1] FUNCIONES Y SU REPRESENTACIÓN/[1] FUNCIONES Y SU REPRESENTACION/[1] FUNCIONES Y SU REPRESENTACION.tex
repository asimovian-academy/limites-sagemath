% Options for packages loaded elsewhere
\PassOptionsToPackage{unicode}{hyperref}
\PassOptionsToPackage{hyphens}{url}
%
\documentclass[
]{article}
\usepackage{amsmath,amssymb}
\usepackage{lmodern}
\usepackage{iftex}
\ifPDFTeX
  \usepackage[T1]{fontenc}
  \usepackage[utf8]{inputenc}
  \usepackage{textcomp} % provide euro and other symbols
\else % if luatex or xetex
  \usepackage{unicode-math}
  \defaultfontfeatures{Scale=MatchLowercase}
  \defaultfontfeatures[\rmfamily]{Ligatures=TeX,Scale=1}
\fi
% Use upquote if available, for straight quotes in verbatim environments
\IfFileExists{upquote.sty}{\usepackage{upquote}}{}
\IfFileExists{microtype.sty}{% use microtype if available
  \usepackage[]{microtype}
  \UseMicrotypeSet[protrusion]{basicmath} % disable protrusion for tt fonts
}{}
\makeatletter
\@ifundefined{KOMAClassName}{% if non-KOMA class
  \IfFileExists{parskip.sty}{%
    \usepackage{parskip}
  }{% else
    \setlength{\parindent}{0pt}
    \setlength{\parskip}{6pt plus 2pt minus 1pt}}
}{% if KOMA class
  \KOMAoptions{parskip=half}}
\makeatother
\usepackage{xcolor}
\IfFileExists{xurl.sty}{\usepackage{xurl}}{} % add URL line breaks if available
\IfFileExists{bookmark.sty}{\usepackage{bookmark}}{\usepackage{hyperref}}
\hypersetup{
  hidelinks,
  pdfcreator={LaTeX via pandoc}}
\urlstyle{same} % disable monospaced font for URLs
\usepackage{color}
\usepackage{fancyvrb}
\newcommand{\VerbBar}{|}
\newcommand{\VERB}{\Verb[commandchars=\\\{\}]}
\DefineVerbatimEnvironment{Highlighting}{Verbatim}{commandchars=\\\{\}}
% Add ',fontsize=\small' for more characters per line
\newenvironment{Shaded}{}{}
\newcommand{\AlertTok}[1]{\textcolor[rgb]{1.00,0.00,0.00}{\textbf{#1}}}
\newcommand{\AnnotationTok}[1]{\textcolor[rgb]{0.38,0.63,0.69}{\textbf{\textit{#1}}}}
\newcommand{\AttributeTok}[1]{\textcolor[rgb]{0.49,0.56,0.16}{#1}}
\newcommand{\BaseNTok}[1]{\textcolor[rgb]{0.25,0.63,0.44}{#1}}
\newcommand{\BuiltInTok}[1]{#1}
\newcommand{\CharTok}[1]{\textcolor[rgb]{0.25,0.44,0.63}{#1}}
\newcommand{\CommentTok}[1]{\textcolor[rgb]{0.38,0.63,0.69}{\textit{#1}}}
\newcommand{\CommentVarTok}[1]{\textcolor[rgb]{0.38,0.63,0.69}{\textbf{\textit{#1}}}}
\newcommand{\ConstantTok}[1]{\textcolor[rgb]{0.53,0.00,0.00}{#1}}
\newcommand{\ControlFlowTok}[1]{\textcolor[rgb]{0.00,0.44,0.13}{\textbf{#1}}}
\newcommand{\DataTypeTok}[1]{\textcolor[rgb]{0.56,0.13,0.00}{#1}}
\newcommand{\DecValTok}[1]{\textcolor[rgb]{0.25,0.63,0.44}{#1}}
\newcommand{\DocumentationTok}[1]{\textcolor[rgb]{0.73,0.13,0.13}{\textit{#1}}}
\newcommand{\ErrorTok}[1]{\textcolor[rgb]{1.00,0.00,0.00}{\textbf{#1}}}
\newcommand{\ExtensionTok}[1]{#1}
\newcommand{\FloatTok}[1]{\textcolor[rgb]{0.25,0.63,0.44}{#1}}
\newcommand{\FunctionTok}[1]{\textcolor[rgb]{0.02,0.16,0.49}{#1}}
\newcommand{\ImportTok}[1]{#1}
\newcommand{\InformationTok}[1]{\textcolor[rgb]{0.38,0.63,0.69}{\textbf{\textit{#1}}}}
\newcommand{\KeywordTok}[1]{\textcolor[rgb]{0.00,0.44,0.13}{\textbf{#1}}}
\newcommand{\NormalTok}[1]{#1}
\newcommand{\OperatorTok}[1]{\textcolor[rgb]{0.40,0.40,0.40}{#1}}
\newcommand{\OtherTok}[1]{\textcolor[rgb]{0.00,0.44,0.13}{#1}}
\newcommand{\PreprocessorTok}[1]{\textcolor[rgb]{0.74,0.48,0.00}{#1}}
\newcommand{\RegionMarkerTok}[1]{#1}
\newcommand{\SpecialCharTok}[1]{\textcolor[rgb]{0.25,0.44,0.63}{#1}}
\newcommand{\SpecialStringTok}[1]{\textcolor[rgb]{0.73,0.40,0.53}{#1}}
\newcommand{\StringTok}[1]{\textcolor[rgb]{0.25,0.44,0.63}{#1}}
\newcommand{\VariableTok}[1]{\textcolor[rgb]{0.10,0.09,0.49}{#1}}
\newcommand{\VerbatimStringTok}[1]{\textcolor[rgb]{0.25,0.44,0.63}{#1}}
\newcommand{\WarningTok}[1]{\textcolor[rgb]{0.38,0.63,0.69}{\textbf{\textit{#1}}}}
\usepackage{graphicx}
\makeatletter
\def\maxwidth{\ifdim\Gin@nat@width>\linewidth\linewidth\else\Gin@nat@width\fi}
\def\maxheight{\ifdim\Gin@nat@height>\textheight\textheight\else\Gin@nat@height\fi}
\makeatother
% Scale images if necessary, so that they will not overflow the page
% margins by default, and it is still possible to overwrite the defaults
% using explicit options in \includegraphics[width, height, ...]{}
\setkeys{Gin}{width=\maxwidth,height=\maxheight,keepaspectratio}
% Set default figure placement to htbp
\makeatletter
\def\fps@figure{htbp}
\makeatother
\setlength{\emergencystretch}{3em} % prevent overfull lines
\providecommand{\tightlist}{%
  \setlength{\itemsep}{0pt}\setlength{\parskip}{0pt}}
\setcounter{secnumdepth}{-\maxdimen} % remove section numbering
\ifLuaTeX
  \usepackage{selnolig}  % disable illegal ligatures
\fi

\author{}
\date{}

\begin{document}

\hypertarget{precuxe1lculo}{%
\section{PRECÁLCULO}\label{precuxe1lculo}}

\hypertarget{funciones-y-sus-luxedmites}{%
\subsection{FUNCIONES Y SUS LÍMITES}\label{funciones-y-sus-luxedmites}}

\hypertarget{juliho-castillo-colmenares}{%
\subsubsection{\texorpdfstring{Juliho Castillo Colmenares
}{Juliho Castillo Colmenares }}\label{juliho-castillo-colmenares}}

\emph{IMPORTANTE: Para correr el siguiente código, necesitarás
\href{https://youtu.be/8KT9GMruOrU}{Sagemath.}}

El cálculo es muy diferente de las matemáticas que has estudiado hasta
hoy, porque tiene que ver con cambios. En esta sección estudiaremos el
comportamiento en cambio de funciones y sus límites.

\hypertarget{funciones-y-su-representaciuxf3n}{%
\subsection{\texorpdfstring{Funciones y su representación.
}{Funciones y su representación. }}\label{funciones-y-su-representaciuxf3n}}

Las funciones surgen cuando una cantidad depende de otra. Consideremos
algunos ejemplos.

\hypertarget{ejemplo-a}{%
\paragraph{Ejemplo A}\label{ejemplo-a}}

El área de un círculo depende del radio del mismo:\\
\( A = \pi r^2 \)

\begin{Shaded}
\begin{Highlighting}[]
\KeywordTok{def}\NormalTok{ area\_circulo(radio):}
    \ControlFlowTok{return} \FloatTok{3.14159}\OperatorTok{*}\NormalTok{radio}\OperatorTok{\^{}}\DecValTok{2}

\BuiltInTok{print}\NormalTok{(area\_circulo(}\FloatTok{2.5267490372}\NormalTok{))}
\end{Highlighting}
\end{Shaded}

\begin{verbatim}
20.0573578810604
\end{verbatim}

\hypertarget{ejemplo-b}{%
\paragraph{Ejemplo B}\label{ejemplo-b}}

La población humana (en todo el mundo) \(P\) depende del tiempo \(t\).
En la siguiente tabla, se muestra el año y la población en milliones
correspondiente:

\begin{Shaded}
\begin{Highlighting}[]
\NormalTok{poblacion }\OperatorTok{=}\NormalTok{ [}
\NormalTok{    [}\DecValTok{1900}\NormalTok{, }\DecValTok{1650}\NormalTok{],}
\NormalTok{    [}\DecValTok{1910}\NormalTok{, }\DecValTok{1750}\NormalTok{],}
\NormalTok{    [}\DecValTok{1920}\NormalTok{, }\DecValTok{1860}\NormalTok{],}
\NormalTok{    [}\DecValTok{1930}\NormalTok{, }\DecValTok{2070}\NormalTok{],}
\NormalTok{    [}\DecValTok{1940}\NormalTok{, }\DecValTok{2300}\NormalTok{],}
\NormalTok{    [}\DecValTok{1950}\NormalTok{, }\DecValTok{2560}\NormalTok{],}
\NormalTok{    [}\DecValTok{1960}\NormalTok{, }\DecValTok{3040}\NormalTok{],}
\NormalTok{    [}\DecValTok{1970}\NormalTok{, }\DecValTok{3710}\NormalTok{],}
\NormalTok{    [}\DecValTok{1980}\NormalTok{, }\DecValTok{4450}\NormalTok{],}
\NormalTok{    [}\DecValTok{1990}\NormalTok{, }\DecValTok{5280}\NormalTok{],}
\NormalTok{    [}\DecValTok{2000}\NormalTok{, }\DecValTok{6080}\NormalTok{],}
\NormalTok{    [}\DecValTok{2010}\NormalTok{, }\DecValTok{6870}\NormalTok{]}
\NormalTok{            ]}
\end{Highlighting}
\end{Shaded}

A partir de esta tabla, podemos crear una función punto a punto:

\begin{Shaded}
\begin{Highlighting}[]
\KeywordTok{def}\NormalTok{ P(t):}
    \ControlFlowTok{for}\NormalTok{ dato }\KeywordTok{in}\NormalTok{ poblacion:}
        \ControlFlowTok{if}\NormalTok{ t}\OperatorTok{==}\NormalTok{dato[}\DecValTok{0}\NormalTok{]:}
            \ControlFlowTok{return}\NormalTok{ dato[}\DecValTok{1}\NormalTok{]}
        
\BuiltInTok{print}\NormalTok{(P(}\DecValTok{1950}\NormalTok{))}
\end{Highlighting}
\end{Shaded}

\begin{verbatim}
2560
\end{verbatim}

\hypertarget{ejemplo-c}{%
\paragraph{Ejemplo C}\label{ejemplo-c}}

El costo \(C\) de enviar un paquete depende del peso \(w\). Por ejemplo,
supongamos que los precio están dados por las siguiente reglas:

\begin{itemize}
\item
  Menor o igual a 10 kgs: Costo fijo de \$150 más \$10 por kilogramo
\item
  Mayor a 10 kgs: \$25 por kilogramo
\item
  La capacidad máxima del envíoi es 20 kgs
\end{itemize}

\begin{Shaded}
\begin{Highlighting}[]
\CommentTok{"""}
\CommentTok{Definimos la función de costo C en función del peso w}
\CommentTok{"""}
\KeywordTok{def}\NormalTok{ C(w):}
    \ControlFlowTok{if}\NormalTok{ w}\OperatorTok{\textless{}}\DecValTok{0}\NormalTok{:}
        \ControlFlowTok{return} \StringTok{"¿Seguro que estás enviando algo?"}
    \ControlFlowTok{elif}\NormalTok{ w}\OperatorTok{\textless{}=}\DecValTok{10}\NormalTok{:}
        \ControlFlowTok{return} \DecValTok{150} \OperatorTok{+} \DecValTok{10}\OperatorTok{*}\NormalTok{w}
    \ControlFlowTok{elif}\NormalTok{ w}\OperatorTok{\textless{}=}\DecValTok{20}\NormalTok{:}
        \ControlFlowTok{return} \DecValTok{25}\OperatorTok{*}\NormalTok{w}
    \ControlFlowTok{else}\NormalTok{:}
        \ControlFlowTok{return} \StringTok{"¡Excediste la capacidad!"}    
\end{Highlighting}
\end{Shaded}

\begin{Shaded}
\begin{Highlighting}[]
\CommentTok{"""}
\CommentTok{Graficamos en el intervalo (0,20)}
\CommentTok{"""}
\ImportTok{import}\NormalTok{ numpy }\ImportTok{as}\NormalTok{ np}
\ImportTok{import}\NormalTok{ matplotlib.pyplot }\ImportTok{as}\NormalTok{ plt}

\NormalTok{x }\OperatorTok{=}\NormalTok{ np.arange(}\DecValTok{0}\NormalTok{,}\DecValTok{20}\NormalTok{,}\FloatTok{0.01}\NormalTok{)}
\NormalTok{C }\OperatorTok{=}\NormalTok{ np.vectorize(C)}
\NormalTok{y }\OperatorTok{=}\NormalTok{ C(x)}

\NormalTok{plt.plot(x,y)}
\NormalTok{plt.show()}
\end{Highlighting}
\end{Shaded}

\begin{figure}
\centering
\includegraphics{C:/Users/julih/Dropbox (Centro Turing)/Repositorios/precalculo-sagemath/{[}1{]} FUNCIONES Y SU REPRESENTACIÓN/{[}1{]} FUNCIONES Y SU REPRESENTACION/output_10_0.png}
\caption{}
\end{figure}

\hypertarget{ejemplo-d}{%
\paragraph{Ejemplo D}\label{ejemplo-d}}

La aceleracion vertical \(a\) del suelo en un terremoto\\
\includegraphics{C:/Users/julih/Dropbox (Centro Turing)/Repositorios/precalculo-sagemath/{[}1{]} FUNCIONES Y SU REPRESENTACIÓN/{[}1{]} FUNCIONES Y SU REPRESENTACION/IM10101.png}

\begin{Shaded}
\begin{Highlighting}[]
\CommentTok{"""}
\CommentTok{Simularemos la acelaración a respecto al tiempo t}
\CommentTok{"""}
\KeywordTok{def}\NormalTok{ a(t):}
    \ControlFlowTok{return}\NormalTok{ np.sin(t)}\OperatorTok{*}\NormalTok{np.random.normal()}
\end{Highlighting}
\end{Shaded}

\begin{Shaded}
\begin{Highlighting}[]
\CommentTok{"""}
\CommentTok{Graficamos en el intervalo (0, pi)}
\CommentTok{"""}
\NormalTok{x }\OperatorTok{=}\NormalTok{ np.arange(}\DecValTok{0}\NormalTok{,np.pi,}\FloatTok{0.01}\NormalTok{)}
\NormalTok{a }\OperatorTok{=}\NormalTok{ np.vectorize(a)}
\NormalTok{y }\OperatorTok{=}\NormalTok{ a(x)}

\NormalTok{plt.plot(x,y)}
\end{Highlighting}
\end{Shaded}

\begin{verbatim}
[<matplotlib.lines.Line2D object at 0x6ffe91c1f1d0>]
\end{verbatim}

\begin{figure}
\centering
\includegraphics{C:/Users/julih/Dropbox (Centro Turing)/Repositorios/precalculo-sagemath/{[}1{]} FUNCIONES Y SU REPRESENTACIÓN/{[}1{]} FUNCIONES Y SU REPRESENTACION/output_13_1.png}
\caption{}
\end{figure}

Una \textbf{función} \(f\) es una regla que asigna a cada elemento \(x\)
de un conjunto \(D\) un elemento \(f(x)\) de un conjunto \(E\).

Al conjunto \(D\) se le llama \textbf{dominio}, mientras que a \(E\) se
le llama \textbf{contradominio}.

\begin{Shaded}
\begin{Highlighting}[]
\NormalTok{x }\OperatorTok{=}\NormalTok{ var(}\StringTok{"x"}\NormalTok{)}
\NormalTok{f(x) }\OperatorTok{=}\NormalTok{ exp(}\OperatorTok{{-}}\NormalTok{x}\OperatorTok{\^{}}\DecValTok{2}\NormalTok{)}\OperatorTok{*}\NormalTok{sin(x)}
\NormalTok{grafica }\OperatorTok{=}\NormalTok{ plot(f)}
\NormalTok{show(grafica)}
\end{Highlighting}
\end{Shaded}

\begin{figure}
\centering
\includegraphics{C:/Users/julih/Dropbox (Centro Turing)/Repositorios/precalculo-sagemath/{[}1{]} FUNCIONES Y SU REPRESENTACIÓN/{[}1{]} FUNCIONES Y SU REPRESENTACION/output_15_0.png}
\caption{}
\end{figure}

\begin{Shaded}
\begin{Highlighting}[]
\NormalTok{grafica }\OperatorTok{=}\NormalTok{ plot(f, (x,}\OperatorTok{{-}}\DecValTok{5}\NormalTok{,}\DecValTok{5}\NormalTok{))}
\NormalTok{show(grafica)}
\end{Highlighting}
\end{Shaded}

\begin{figure}
\centering
\includegraphics{C:/Users/julih/Dropbox (Centro Turing)/Repositorios/precalculo-sagemath/{[}1{]} FUNCIONES Y SU REPRESENTACIÓN/{[}1{]} FUNCIONES Y SU REPRESENTACION/output_16_0.png}
\caption{}
\end{figure}

\hypertarget{representaciuxf3n-de-funciones}{%
\subsubsection{Representación de
funciones}\label{representaciuxf3n-de-funciones}}

\hypertarget{ejercicio-1}{%
\paragraph{Ejercicio}\label{ejercicio-1}}

Encuentre el dominio de cada función:

\begin{itemize}
\item
  \(f(x)=\sqrt{x+2}\)
\item
  \(g(x)=\dfrac{1}{x^2-x}\)
\end{itemize}

\hypertarget{prueba-de-la-linea-vertical}{%
\paragraph{\texorpdfstring{Prueba de la linea vertical
}{Prueba de la linea vertical }}\label{prueba-de-la-linea-vertical}}

Una curva en el plano \(xy\) es la gráfica de una función \(y=f(x)\) si
y solo si no existe línea vertical alguna que intersecte la curva más de
dos veces.

\hypertarget{ejemplo-1}{%
\paragraph{Ejemplo}\label{ejemplo-1}}

Determina si la ecuación \(x^2-y^2=1\) define una función \(y=f(x)\).

\begin{Shaded}
\begin{Highlighting}[]
\NormalTok{x,y }\OperatorTok{=}\NormalTok{ var(}\StringTok{"x,y"}\NormalTok{)}
\NormalTok{ecuacion }\OperatorTok{=}\NormalTok{ x}\OperatorTok{\^{}}\DecValTok{2}\OperatorTok{{-}}\NormalTok{y}\OperatorTok{\^{}}\DecValTok{2}\OperatorTok{==}\DecValTok{1}
\NormalTok{grafica }\OperatorTok{=}\NormalTok{ implicit\_plot(ecuacion, (x,}\OperatorTok{{-}}\DecValTok{2}\NormalTok{,}\DecValTok{2}\NormalTok{), (y,}\OperatorTok{{-}}\DecValTok{2}\NormalTok{,}\DecValTok{2}\NormalTok{))}
\NormalTok{show(grafica)}
\end{Highlighting}
\end{Shaded}

\begin{figure}
\centering
\includegraphics{C:/Users/julih/Dropbox (Centro Turing)/Repositorios/precalculo-sagemath/{[}1{]} FUNCIONES Y SU REPRESENTACIÓN/{[}1{]} FUNCIONES Y SU REPRESENTACION/output_21_0.png}
\caption{}
\end{figure}

\hypertarget{ejemplo-2}{%
\paragraph{Ejemplo}\label{ejemplo-2}}

Determina si la ecuación \((x-1)^2=4(y-2)\) define una función
\(y=f(x)\).

\begin{Shaded}
\begin{Highlighting}[]
\NormalTok{x,y }\OperatorTok{=}\NormalTok{ var(}\StringTok{"x,y"}\NormalTok{)}
\NormalTok{ecuacion }\OperatorTok{=}\NormalTok{ (x}\OperatorTok{{-}}\DecValTok{1}\NormalTok{)}\OperatorTok{\^{}}\DecValTok{2}\OperatorTok{==}\DecValTok{4}\OperatorTok{*}\NormalTok{(y}\OperatorTok{{-}}\DecValTok{2}\NormalTok{)}
\NormalTok{grafica }\OperatorTok{=}\NormalTok{ implicit\_plot(ecuacion, (x,}\DecValTok{0}\NormalTok{,}\DecValTok{2}\NormalTok{), (y,}\DecValTok{1}\NormalTok{,}\DecValTok{3}\NormalTok{))}
\NormalTok{show(grafica)}
\end{Highlighting}
\end{Shaded}

\begin{figure}
\centering
\includegraphics{C:/Users/julih/Dropbox (Centro Turing)/Repositorios/precalculo-sagemath/{[}1{]} FUNCIONES Y SU REPRESENTACIÓN/{[}1{]} FUNCIONES Y SU REPRESENTACION/output_23_0.png}
\caption{}
\end{figure}

\hypertarget{ejemplo-3}{%
\paragraph{\texorpdfstring{Ejemplo }{Ejemplo }}\label{ejemplo-3}}

Determina si la gráfica de las ecuaciones paramétricas:\\
\( x = 10t, y = 50+5t-\dfrac{1}{2}(9.8)t^2\)\\
define una función \(y=f(x)\).

\begin{Shaded}
\begin{Highlighting}[]
\NormalTok{t }\OperatorTok{=}\NormalTok{ var(}\StringTok{"t"}\NormalTok{)}
\NormalTok{x(t) }\OperatorTok{=} \DecValTok{10}\OperatorTok{*}\NormalTok{t}
\NormalTok{y(t) }\OperatorTok{=} \DecValTok{50}\OperatorTok{+}\DecValTok{5}\OperatorTok{*}\NormalTok{t}\OperatorTok{{-}}\NormalTok{(}\DecValTok{1}\OperatorTok{/}\DecValTok{2}\NormalTok{)}\OperatorTok{*}\NormalTok{(}\FloatTok{9.8}\NormalTok{)}\OperatorTok{*}\NormalTok{t}\OperatorTok{\^{}}\DecValTok{2}
\NormalTok{grafica }\OperatorTok{=}\NormalTok{ parametric\_plot((x(t),y(t)), (t,}\DecValTok{0}\NormalTok{,}\DecValTok{1}\NormalTok{))}

\NormalTok{show(grafica)}
\end{Highlighting}
\end{Shaded}

\begin{figure}
\centering
\includegraphics{C:/Users/julih/Dropbox (Centro Turing)/Repositorios/precalculo-sagemath/{[}1{]} FUNCIONES Y SU REPRESENTACIÓN/{[}1{]} FUNCIONES Y SU REPRESENTACION/output_25_0.png}
\caption{}
\end{figure}

\hypertarget{ejercicio-de-exploraciuxf3n-1}{%
\paragraph{Ejercicio de
exploración}\label{ejercicio-de-exploraciuxf3n-1}}

Las funciones anteriores representan la posición de un proyectil en
función del tiempo. Determina gráficamente:

\begin{itemize}
\item
  la distancia a la que choca contra el suelo
\item
  cuanto tiempo tarde en ocurrir esto
\item
  la altura máxima y el tiempo en que se alcanza
\end{itemize}

¿Puedes determinar cuál es la función \(y=f(x)\)?

\hypertarget{ejercicio-de-exploraciuxf3n-2}{%
\paragraph{Ejercicio de
exploración}\label{ejercicio-de-exploraciuxf3n-2}}

Muchas relaciones fundamentales en matemáticas y física no se pueden
expresar como funciones. Aún así se puede definir el dominio de una
relación. Averigua en internet como es que se hace esto y determinalo
gráficamente para la relación

\( \dfrac{(x-2)^2}{9} + \dfrac{(y-3)^2}{4} = 1. \)

\emph{¿Cuál es el intervalo minimal en el eje \(y\) que necesitas para
graficar completamente la relación?}

\hypertarget{funciones-representadas-a-trozos}{%
\subsubsection{Funciones representadas a
trozos}\label{funciones-representadas-a-trozos}}

\hypertarget{ejemplo-4}{%
\paragraph{Ejemplo}\label{ejemplo-4}}

Una función \(f\) esta definida por\\
\(f(x)=
\begin{cases}
1-x & x \leq 1\\
x^2 & x>1
\end{cases}
\)

Evaluar \(f(0), f(1), f(2)\) y graficar la función.

\begin{Shaded}
\begin{Highlighting}[]
\KeywordTok{def}\NormalTok{ f(x):}
    \ControlFlowTok{if}\NormalTok{ x}\OperatorTok{\textless{}=}\DecValTok{1}\NormalTok{:}
        \ControlFlowTok{return} \DecValTok{1}\OperatorTok{{-}}\NormalTok{x}
    \ControlFlowTok{else}\NormalTok{:}
        \ControlFlowTok{return}\NormalTok{ x}\OperatorTok{**}\DecValTok{2}
\end{Highlighting}
\end{Shaded}

\begin{Shaded}
\begin{Highlighting}[]
\ImportTok{import}\NormalTok{ numpy }\ImportTok{as}\NormalTok{ np}
\NormalTok{f }\OperatorTok{=}\NormalTok{ np.vectorize(f)}
\NormalTok{x }\OperatorTok{=}\NormalTok{ [}\DecValTok{0}\NormalTok{,}\DecValTok{1}\NormalTok{,}\DecValTok{2}\NormalTok{]}
\NormalTok{y }\OperatorTok{=}\NormalTok{ f(x)}
\BuiltInTok{print}\NormalTok{(y)}
\end{Highlighting}
\end{Shaded}

\begin{verbatim}
[1 0 4]
\end{verbatim}

\begin{Shaded}
\begin{Highlighting}[]
\ImportTok{import}\NormalTok{ matplotlib.pyplot }\ImportTok{as}\NormalTok{ plt}

\NormalTok{x }\OperatorTok{=}\NormalTok{ np.arange(}\DecValTok{0}\NormalTok{,}\DecValTok{2}\NormalTok{,}\FloatTok{0.001}\NormalTok{)}
\NormalTok{y }\OperatorTok{=}\NormalTok{ f(x)}

\NormalTok{plt.plot(x,y)}
\NormalTok{plt.show()}
\end{Highlighting}
\end{Shaded}

\begin{figure}
\centering
\includegraphics{C:/Users/julih/Dropbox (Centro Turing)/Repositorios/precalculo-sagemath/{[}1{]} FUNCIONES Y SU REPRESENTACIÓN/{[}1{]} FUNCIONES Y SU REPRESENTACION/output_32_0.png}
\caption{}
\end{figure}

\hypertarget{ejercicio-valor-absoluto}{%
\paragraph{Ejercicio (Valor absoluto)}\label{ejercicio-valor-absoluto}}

\begin{itemize}
\item
  Investiga como se define la función \emph{valor absoluto}
\item
  Implementala en \texttt{Python}
\item
  Evaluala en el arreglo \(x=-1,0,1\)
\item
  Traza la gráfica
\end{itemize}

\hypertarget{simetruxeda}{%
\subsubsection{Simetría}\label{simetruxeda}}

Si una función \(f(x)\) satisface la relación \(f(x)=f(-x)\) en su
dominio, diremos que la función es \textbf{par}. Su gráfica es simétrica
respecto al eje \(y\).

\hypertarget{ejemplo-5}{%
\paragraph{Ejemplo}\label{ejemplo-5}}

\( f(x) = x^2 \)

\begin{Shaded}
\begin{Highlighting}[]
\NormalTok{x }\OperatorTok{=}\NormalTok{ var(}\StringTok{"x"}\NormalTok{)}
\NormalTok{f(x) }\OperatorTok{=}\NormalTok{ x}\OperatorTok{\^{}}\DecValTok{2}
\NormalTok{show(f(}\OperatorTok{{-}}\NormalTok{x))}
\end{Highlighting}
\end{Shaded}

\textbackslash newcommand\{\textbackslash Bold\}{[}1{]}\{\textbackslash mathbf\{\#1\}\}x\^{}\{2\}

\begin{Shaded}
\begin{Highlighting}[]
\NormalTok{plot(f)}
\end{Highlighting}
\end{Shaded}

\begin{figure}
\centering
\includegraphics{C:/Users/julih/Dropbox (Centro Turing)/Repositorios/precalculo-sagemath/{[}1{]} FUNCIONES Y SU REPRESENTACIÓN/{[}1{]} FUNCIONES Y SU REPRESENTACION/output_37_0.png}
\caption{}
\end{figure}

\hypertarget{ejemplo-6}{%
\paragraph{Ejemplo}\label{ejemplo-6}}

\( g(x) = cos(x) \)

\begin{Shaded}
\begin{Highlighting}[]
\NormalTok{x }\OperatorTok{=}\NormalTok{ var(}\StringTok{"x"}\NormalTok{)}
\NormalTok{g(x) }\OperatorTok{=}\NormalTok{ cos(x)}
\NormalTok{show(g(}\OperatorTok{{-}}\NormalTok{x).simplify())}
\end{Highlighting}
\end{Shaded}

\textbackslash newcommand\{\textbackslash Bold\}{[}1{]}\{\textbackslash mathbf\{\#1\}\}\textbackslash cos\textbackslash left(x\textbackslash right)

\begin{Shaded}
\begin{Highlighting}[]
\NormalTok{plot(g, (x,}\OperatorTok{{-}}\DecValTok{2}\OperatorTok{*}\NormalTok{pi, }\DecValTok{2}\OperatorTok{*}\NormalTok{pi))}
\end{Highlighting}
\end{Shaded}

\begin{figure}
\centering
\includegraphics{C:/Users/julih/Dropbox (Centro Turing)/Repositorios/precalculo-sagemath/{[}1{]} FUNCIONES Y SU REPRESENTACIÓN/{[}1{]} FUNCIONES Y SU REPRESENTACION/output_40_0.png}
\caption{}
\end{figure}

Si una función \(f(x)\) satisface la relación \(f(-x)=-f(x)\) en su
dominio, diremos que la función es \textbf{impar}. Su gráfica es
simétrica respecto al origen.

\hypertarget{ejemplo-4}{%
\paragraph{Ejemplo 4}\label{ejemplo-4}}

\(f(x) = x^3\)

\begin{Shaded}
\begin{Highlighting}[]
\NormalTok{x }\OperatorTok{=}\NormalTok{ var(}\StringTok{"x"}\NormalTok{)}
\NormalTok{f(x) }\OperatorTok{=}\NormalTok{ x}\OperatorTok{\^{}}\DecValTok{3}
\NormalTok{show(f(}\OperatorTok{{-}}\NormalTok{x).simplify())}
\end{Highlighting}
\end{Shaded}

\textbackslash newcommand\{\textbackslash Bold\}{[}1{]}\{\textbackslash mathbf\{\#1\}\}-x\^{}\{3\}

\begin{Shaded}
\begin{Highlighting}[]
\NormalTok{plot(f)}
\end{Highlighting}
\end{Shaded}

\begin{figure}
\centering
\includegraphics{C:/Users/julih/Dropbox (Centro Turing)/Repositorios/precalculo-sagemath/{[}1{]} FUNCIONES Y SU REPRESENTACIÓN/{[}1{]} FUNCIONES Y SU REPRESENTACION/output_44_0.png}
\caption{}
\end{figure}

\hypertarget{ejemplo-7}{%
\paragraph{\texorpdfstring{Ejemplo }{Ejemplo }}\label{ejemplo-7}}

\(g(x) = \sin(x)\)

\begin{Shaded}
\begin{Highlighting}[]
\NormalTok{x }\OperatorTok{=}\NormalTok{ var(}\StringTok{"x"}\NormalTok{)}
\NormalTok{g(x) }\OperatorTok{=}\NormalTok{ sin(x)}
\NormalTok{show(g(}\OperatorTok{{-}}\NormalTok{x).simplify())}
\end{Highlighting}
\end{Shaded}

\textbackslash newcommand\{\textbackslash Bold\}{[}1{]}\{\textbackslash mathbf\{\#1\}\}-\textbackslash sin\textbackslash left(x\textbackslash right)

\begin{Shaded}
\begin{Highlighting}[]
\NormalTok{plot(g, (x,}\OperatorTok{{-}}\NormalTok{pi,pi))}
\end{Highlighting}
\end{Shaded}

\begin{figure}
\centering
\includegraphics{C:/Users/julih/Dropbox (Centro Turing)/Repositorios/precalculo-sagemath/{[}1{]} FUNCIONES Y SU REPRESENTACIÓN/{[}1{]} FUNCIONES Y SU REPRESENTACION/output_47_0.png}
\caption{}
\end{figure}

\hypertarget{ejercicio-2}{%
\paragraph{Ejercicio}\label{ejercicio-2}}

Determina si las siguiente funciones son pares, impares o ninguna:

\begin{itemize}
\item
  \(f(x)=x^5+x\)
\item
  \(g(x)=1-x^4\)
\item
  \(h(x)=2x-x^2\)
\end{itemize}

\hypertarget{funciones-crecientes-y-decrecientes}{%
\paragraph{Funciones crecientes y
decrecientes}\label{funciones-crecientes-y-decrecientes}}

Una función \(f\) es \textbf{creciente} en un intervalo \(I\) si\\
\(f(x_0) < f(x_1)\) siempre que \(x_1 < x_2\).

Una función \(f\) es \textbf{decreciente} en un intervalo \(I\) si\\
\(f(x_0) < f(x_1)\) siempre que \(x_1 < x_2\).

\textbf{Observación} El intervalo es muy importante para determinar si
la función es creciente o no. Consideremos la función \(f(x)=x^2\) en
los intervalos

\begin{itemize}
\item
  \(I=(-1,0)\)
\item
  \(I=(0,1)\)
\item
  \(I=(-1,-1)\)
\end{itemize}

\begin{Shaded}
\begin{Highlighting}[]
\NormalTok{x }\OperatorTok{=}\NormalTok{ var(}\StringTok{"x"}\NormalTok{)}
\NormalTok{f(x) }\OperatorTok{=}\NormalTok{ x}\OperatorTok{\^{}}\DecValTok{2}
\end{Highlighting}
\end{Shaded}

\begin{Shaded}
\begin{Highlighting}[]
\NormalTok{plot(f, (x,}\OperatorTok{{-}}\DecValTok{1}\NormalTok{,}\DecValTok{0}\NormalTok{))}
\end{Highlighting}
\end{Shaded}

\begin{figure}
\centering
\includegraphics{C:/Users/julih/Dropbox (Centro Turing)/Repositorios/precalculo-sagemath/{[}1{]} FUNCIONES Y SU REPRESENTACIÓN/{[}1{]} FUNCIONES Y SU REPRESENTACION/output_52_0.png}
\caption{}
\end{figure}

\begin{Shaded}
\begin{Highlighting}[]
\NormalTok{plot(f, (x,}\DecValTok{0}\NormalTok{,}\DecValTok{1}\NormalTok{))}
\end{Highlighting}
\end{Shaded}

\begin{figure}
\centering
\includegraphics{C:/Users/julih/Dropbox (Centro Turing)/Repositorios/precalculo-sagemath/{[}1{]} FUNCIONES Y SU REPRESENTACIÓN/{[}1{]} FUNCIONES Y SU REPRESENTACION/output_53_0.png}
\caption{}
\end{figure}

\begin{Shaded}
\begin{Highlighting}[]
\NormalTok{plot(f, (x,}\OperatorTok{{-}}\DecValTok{1}\NormalTok{,}\DecValTok{1}\NormalTok{))}
\end{Highlighting}
\end{Shaded}

\begin{figure}
\centering
\includegraphics{C:/Users/julih/Dropbox (Centro Turing)/Repositorios/precalculo-sagemath/{[}1{]} FUNCIONES Y SU REPRESENTACIÓN/{[}1{]} FUNCIONES Y SU REPRESENTACION/output_54_0.png}
\caption{}
\end{figure}

\end{document}
